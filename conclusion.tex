
Recent advances in manufacturing techniques have enabled controlling energy transfer in nanoscale structures. These advances have both improved the efficiency of existing device applications such as thermoelectric heat-to-electricity converters and spawned new applications such as heat-assisted magnetic recording. Releasing the full potential of thermal energy requires, however, a more thorough understanding of energy transfer and conversion in nanoscale.

% Control of charge transfer in small structures, electronics, has given rise to the Digital Revolution.

In this thesis, we have developed new computational methods and analysis procedures for investigating energy transfer in nanoscale systems.  We developed a mathematical expression for the spectral decomposition of the thermal current, used in conjunction with non-equilibrium molecular dynamics simulation to investigate non-linear phononic energy transfer at interfaces and in nanotubes. We also showed that Langevin theory and the Green's function solution of equations of motion can be used to determine phononic, photonic as well as electronic energy transfer rates using very similar calculation methods, unifying the mathematical treatment of different energy carriers. %The methods were applied along with existing methods to investigate phononic and photonic energy transfer. % deliver new physical insight into the complex energy-related processes taking place in nanoscale structures. 

Using non-equilibrium molecular dynamics, we showed that anharmonic phonon-phonon scattering at a material interface increases the interfacial conductance by providing a dissipation mechanism for evanescent waves and enabling multi-phonon energy transfer processes at the interface. As material interfaces typically form the major bottleneck for heat flow in electronic devices, such a detailed understanding of energy transfer might assist in designing better thermally conducting interfaces in future. We also provided numerical evidence that energy transfer in carbon nanotubes is partially ballistic even in micrometer-long tubes due to the long mean free paths at low vibrational frequencies, providing further support for the applicability of carbon nanotubes in thermal management. In silicon nanowires, periodic twinning with a suitably chosen twinning period was shown to strongly decrease the thermal conductivity, suggesting that the thermoelectric conversion efficiency of nanowires could be further increased by such nanostructuring. 

Green's function methods were used to investigate thermal conduction through point contacts in two-dimensional lattices, highlighting the importance of quantum statistics in determining the local temperature. For a point contact in graphene at room temperature, quantum statistical and classical calculations deliver, however, very similar temperature profiles, suggesting that the low-frequency modes (not affected by quantum statistics) are primarily responsible for energy transfer in such structures. Green's function solution of Langevin equations of motion was also applied to investigating electromagnetic energy transfer between polar nanoparticles in a mirror cavity, proposing the possibility to tune electromagnetic energy transfer rates by the modification of electromagnetic density of states. 

The spectral heat current formula developed in this thesis holds great promise for future investigations of energy transfer in nanoscale. Combined with the versatility of molecular dynamics simulations, the spectral decomposition can present a detailed picture of energy transfer processes in various systems, including, e.g., the interface between a solid and a liquid, important in applications such as hyperthermic treatment by heated nanoparticles. Determining the mean free paths is expected to provide theoretical supplement to experimental mean free path spectroscopy measurements and could also provide microscopic insight to the anomalous thermal conduction observed in low-dimensional systems.

Green's function modeling of energy transfer has a lot of potential due to its applicability to phononic, photonic as well as electronic systems. The method can account for quantum statistics, wave effects such as interference, and even carrier dissipation in terms of relaxation rates. Energy exchange between carriers, necessary in the modeling of, e.g., electronic generation of heat in nanoscale structures, could be flexibly modeled using the Green's function method by including an energy exchange term between the carriers in the thermal balance equations. At first order, this energy exchange is expected to be given by the product of the carrier-carrier coupling coefficient and the temperature difference between different systems.

\textbf{FINAL REMARKS ON FUTURE}

% Using NEMD, we investigated phononic energy transfer at an interface between two crystals, carbon nanotube, silicon nanowire with periodic twinning, and point contact in a two-dimensional lattice. 

% Short summary of results

% Outlook
% - Spectral analysis of energy current: liquids and gases, engineering of thermal conductivity in non-linear structures, new understanding of thermal rectification, investigation of one-dimensional thermal conduction, determining mean free paths in amorphous and crystalline materials
% - Green's function methods: self-consistent temperature calculation at interfaces, coupling of different models