
Rapid advances in material synthesis and processing techniques have enabled controlling the flow of heat in nanometric scale. These advances are expected to both improve the efficiency of existing device applications such as thermoelectric heat-to-electricity converters and to spawn new applications such as heat-assisted magnetic recording. Releasing the full potential of thermal energy and perfecting the engineering capabilities requires, however, greater scientific understanding of energy transfer and conversion in nanoscale.

In this thesis, we have developed new computational methods and analysis procedures for investigating energy transfer in nanoscale systems. This work has lead, e.g., to a method to calculate and analyze the spectral heat current, providing access to complete spectral information of thermal transfer in non-equilibrium systems. We have also showed that Langevin theory combined with Green's function techniques allows for determining phononic, photonic as well as electronic energy transfer rates using very similar calculation methods, unifying the mathematical treatment of different energy carriers. % We have applied the method to investigate non-linear phononic energy transfer at interfaces and in nanotubes. 

Using non-equilibrium molecular dynamics, we have showed that anharmonic phonon-phonon scattering at a material interface and its vicinity increases the interfacial conductance by providing a dissipation mechanism for evanescent waves and enabling multi-phonon energy transfer processes at the interface. As material interfaces typically form the major bottleneck for heat flow in electronic devices, such a detailed understanding of energy transfer might assist in designing better thermally conducting interfaces in the future. We have also provided numerical evidence that energy transfer in carbon nanotubes is partially ballistic even in micrometer-long tubes due to the long phonon mean free paths at low vibrational frequencies, providing further evidence on the suitability of carbon nanotubes for thermal management applications. In silicon nanowires, periodic twinning with a suitably chosen twinning period was shown to strongly decrease the thermal conductivity, suggesting that the thermoelectric conversion efficiency of nanowires could be further increased by such nanostructuring. 

GF methods were used to investigate thermal conduction through point contacts in two-dimensional lattices, highlighting the potential importance of quantum statistics in determining the local temperature. For a point contact in graphene at room temperature, quantum statistical and classical calculations deliver, however, very similar temperature profiles, suggesting that the low-frequency modes (not affected by quantum statistics) are primarily responsible for energy transfer in such structures. In future, it would be fruitful to extend the presented phonon transport model by adding electron transport and electron-phonon coupling in the model. Coupling could be introduced by allowing for energy transfer between the electron and phonon baths at each atomic site, analogously to the two-temperature model \cite{chen01}. Such a coupled model might turn out to be very useful in investigations of electronic heating in nanostructures, where the wave effects in electron and phonon transport cannot be neglected. Green's function solution of Langevin equations of motion was also applied in the thesis to investigating electromagnetic energy transfer between polar nanoparticles in a mirror cavity, proposing the possibility to tune electromagnetic energy transfer rates by the modification of electromagnetic density of states. 

The spectral heat current formula developed in this thesis provides a new paradigm for future investigations of energy transfer in nanoscale. Combined with the versatility of molecular dynamics simulations, the spectral decomposition can present a detailed picture of energy transfer processes in complex systems. One potential application is to investigate the heat transfer mechanisms at the interface between a solid and a liquid, important in applications such as hyperthermic tumor treatment. Determining phonon mean free paths in different materials using the non-equilibrium methods developed in this thesis is expected to provide theoretical supplement to experimental mean free path measurements and could also provide microscopic insight to the anomalous thermal conduction observed in low-dimensional systems.

Future progress in miniaturization techniques will provide novel tools for controlling energy flow using nanoscale structures, potentially enabling clean production of energy from waste heat, reducing the energy consumption of digital electronics all around us, and revolutionizing the future by completely new technologies. Feasible realization of such applications is intimately linked with our scientific understanding of nanoscale energy transfer, created by the interplay of theoretical, computational, and experimental research. This thesis contributes to the scientific knowledge by providing new paradigms and methods to modeling energy transfer in atomic-scale structures, potentially playing a role in leading the humanity to a more sustainable way of life. 
