
Recent advances in manufacturing techniques have enabled controlling energy transfer in nanoscale structures. These advances have both improved the efficiency of existing device applications such as thermoelectric heat-to-electricity converters and spawned new applications such as heat-assisted magnetic recording. Releasing the full potential of thermal energy requires, however, a more thorough understanding of energy transfer and conversion in nanoscale.

% Control of charge transfer in small structures, electronics, has given rise to the Digital Revolution.

In this thesis, we have developed new computational methods and applied these methods along with existing methods to investigate energy transfer in different structures. We developed a mathematical expression for the spectral decomposition of the thermal current, used in conjunction with non-equilibrium molecular dynamics (NEMD) simulation to investigate non-linear phononic energy transfer at interfaces and in nanotubes. We also showed in this summary part how Langevin theory and the Green's function solution of equations of motion can be used to determine phononic, photonic as well as electronic energy transfer rates using very similar calculation methods, unifying the mathematical treatment of different energy carriers. % deliver new physical insight into the complex energy-related processes taking place in nanoscale structures. 

Using NEMD, we showed that anharmonic phonon-phonon scattering at a material interface increases the interfacial conductance by providing a dissipation mechanism for evanescent waves and enabling multi-phonon energy transfer processes at the interface. As material interfaces typically form the major bottleneck for heat flow in electronic devices, such a detailed understanding of energy transfer might assist in designing better thermally conducting interfaces in future. We also provided numerical evidence that energy transfer in micrometer-long carbon nanotubes is still partially ballistic due to the long mean free paths at low vibrational frequencies. In silicon nanowires, periodic twinning with a suitably chosen twinning period was shown to strongly decrease the thermal conductivity, suggesting that the thermoelectric conversion efficiency of nanowires could be further increased by such nanostructuring. 



% Using NEMD, we investigated phononic energy transfer at an interface between two crystals, carbon nanotube, silicon nanowire with periodic twinning, and point contact in a two-dimensional lattice. 

% Short summary of results

% Outlook
% - Spectral analysis of energy current: liquids and gases, engineering of thermal conductivity in non-linear structures, new understanding of thermal rectification, investigation of one-dimensional thermal conduction, determining mean free paths in amorphous and crystalline materials
% - Green's function methods: self-consistent temperature calculation at interfaces, coupling of different models