\section{Langevin theory and Green's functions}

Energy transfer rates are calculated by writing down the equations of motion and solving for thermal properties. In many cases, the equations describing the system's dynamics reduce to linear non-homogeneous equations. In the context of lattice heat transfer, this is the case if (i) the anharmonic terms in the lattice dynamics equations (XXX) can be neglected due to, e.g., low temperature or (ii) the anharmonic interactions are represented by an effective, linear coupling to a thermal bath. In electromagnetic energy transfer, the Maxwell equations are directly linear if the material parameters such as the dielectric permittivity are independent of the field strength, which is the case in the materials considered in this thesis. 

\subsection{General theory}

As stated above, the equations of motion for the dynamical degrees of freedom $\bu(\omega)$ are linear in $\bu(\omega)$. The degrees of freedom can be, e.g., the atomic displacement, dipole moment, or the electron annihilation operator. By rearranging and by performing temporal Fourier transform to eliminate the time-derivatives, the equation of motion is generally of the form
\begin{equation}
 [\bb{A}_0+\Sigma(\omega)] \hat\bu(\omega) =  \hat \xi(\omega).
\end{equation}
The coefficient matrix $\bb{A}_0(\omega)$ describes the dynamics and the coupling between the different degrees of freedom of the system. The fluctuations and dissipation arising from each bath $I\in \{1,\dots,N\}$ appear in the equation of motion through the self-energies $\Sigma(\omega)=\sum_I \Sigma_I(\omega)$ and the stochastic forces $\xi(\omega)=\sum_I \xi_I(\omega)$. The fluctuation-dissipation relation connecting the two processes can be written in a compact form in terms of the bath coupling function $\Gamma(\omega)=-2\textrm{Im}[\Sigma(\omega)]$:
\begin{equation}
 \langle \hat \xi_I(\omega)\hat \xi_J(\omega')^T \rangle = 2\pi \hbar \delta(\omega+\omega')\delta_{IJ} \Gamma_I(\omega) \left[f_B(\omega,T_I)+\frac{1}{2} \right].
\end{equation}
Here $T_I$ is the bath temperature.

By defining the Green's function as the inverse matrix of the coefficient matrix $\bb{A}(\omega)$ as $\bG(\omega)= -\bb{A}(\omega)$ (the minus sign is conventional), one can write down the solution to the equation of motion as
\begin{equation}
 \hat\bu(\omega) = - \bb{G}(\omega)\hat \xi(\omega).
\end{equation}
Due to the interactions with the bath, the Green's function $\bb{G}(\omega)$ is modified from its free value $\bb{G}^0(\omega)=-\bb{A}_0\pom ^{-1}$ according to the Dyson equation:
\begin{equation}
 \bb{G}(\omega) ^{-1} = \bb{G}^0(\omega)^{-1}-\sum_I \Sigma_I(\omega).
\end{equation}

Having the solution \eqref{} and the fluctuation-dissipation theorem available, one can calculate the expectation value of any observable (expressible in terms of $\bu$ and $\xi$). Typically we are interested in the energy current $\langle Q_I \rangle$ into the thermal bath. While details for different systems are given below, the currents in all cases reduce to the Landauer-B\"uttiker form
\begin{equation}
 \langle Q_I \rangle = \int_0^{\infty} \frac{d\omega}{2\pi} \hbar \omega \sum_J \ca{T}_{IJ}(\omega) \left[ f_B(\omega,T_J)-f_B(\omega,T_I)\right],
\end{equation}
where the energy transmission function between baths $I$ and $J$ is given by the Caroli form
\begin{equation}
 \ca{T}_{IJ}(\omega) = \textrm{Tr} \left[ \bb{G}(\omega) \Gamma_I(\omega) \bb{G}(\omega)^{\dagger} \Gamma_J(\omega) \right].
\end{equation}



% The linearity of the equations then allows for writing the down the solution for arbitrary source terms in terms of the Green's function. The source terms arise from the coupling to external reservoirs (or to self-consistent thermal baths, see below). Below, we outline the solution for the equations of motion and the conductance calculations for a vibrating lattice and electromagnetically interacting oscillating dipoles.

%Green's functions arise naturally when solving linear non-homogeneous equations. If the linear equation is solved with so-called unit-impulse as the source term, the solution can be used to write down the solution for 

\iffalse
In physics, the Green's function method has two customary meanings. In its more general use, Green's function method refers to solving linear equations of motion in terms of the mathematical Green's function. This is the meaning implied in this work. In quantum mechanical many-body systems, on the other hand, the quantum-mechanical propagators appearing in the diagrammatic perturbation theory are also referred to as Green's functions. The non-equilibrium Keldysh Green's function formalism is briefly described in App. \ref{app:keldysh}. 

To introduce the classical Green's functions, we consider a general non-homogenous equation of the form
 \begin{equation}
  \mathcal{L} f = g, \label{eq:Lf=g}
 \end{equation}
 where $\mathcal{L}$ is a linear operator and $g$ is the source function. The symbolic solution to Eq. \eqref{eq:Lf=g} in terms of the Green's function $\mathcal{G}$ is
 \begin{equation}
  f = -\mathcal{G} g.
 \end{equation}
where the Green's function $\mathcal{G}$ is defined as the inverse of $\mathcal{L}$ (minus sign appears for later convenience):
 \begin{equation}
  \mathcal{L} \mathcal{G} = -I.
 \end{equation}
 Here $I$ is the identity operator. Since $\mathcal{L}$ is linear, solution for 
 \begin{equation}
 \mathcal{L} f = g_1 + g_2
 \end{equation}
 is the sum of solutions
 \begin{equation}
  f = \mathcal{G}g_1 + \mathcal{G} g_2.
 \end{equation}
 Calculating $\mathcal{G}$ for a given $\mathcal{L}$ determines, therefore, the solution for any source function $g$. 

We consider two problems in which the Green's function is used to calculate heat transfer rates: (i) thermal conduction through a nanojunction (phonon heat transfer) and (ii) electromagnetic energy transfer between fluctuating dipoles (photon heat transfer).
\fi
\subsection{Self-consistent heat bath model}



\iffalse
To derive an algebraic expression for $\bb{u}_C$ in terms of the source terms, we first eliminate the time derivatives by performing temporal Fourier transform:
\begin{equation}
  -m\omega^2 \tilde{\bb{u}}_I = - \bb{K}_I \tilde{\bb{u}}_I - \sum_{J\neq I} \bb{V}_{IJ} \tilde{\bb{u}}_J +i m \gamma_I \omega \tilde{\bb{u}}_I + \tilde{\xi}_I.
\end{equation}
Solving for $\tilde{\bb{u}}_I$, we get
\begin{equation}
 \bb{u}_I(\omega) = \bb{g}_I(\omega) \left[ \sum_{J\neq I} \bb{V}_{IJ} \hat{\bb{u}}_J(\omega) \right]
\end{equation}
\fi



\subsection{Electromagnetic heat transfer}
