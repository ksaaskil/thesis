
\chapter{General properties of Langevin heat baths} 

\section{Background}
In his seminal work on the theory of Brownian motion, Paul Langevin added stochastic force terms in the equation of motion to model the essentially random collisions of a particle with the molecules of the surrounding fluid. The additional force consists of two terms, the deterministic damping force proportional to the friction coefficient $\gamma$ and the stochastic force $\xi$:
\begin{equation}
 m \ddot{x} = {F}[{x}(t)] -m \gamma \dot{{x}} + \xi(t). \label{eq:langevin_eq}
\end{equation}
Here ${x}(t)$ is the particle position, $m$ the mass and ${F}[{x}(t)]$ is the force due to particles other than the solvent. For simplicity, we have written the one-dimensional form of the equation. In Langevin theory, the collisions with the solvent (represented by the stochastic force $\xi$) are assumed to average to zero force ($\langle \xi \rangle=0$) and to be temporally uncorrelated: $\langle \xi(t) \xi(t')^T\rangle \propto \delta(t-t')$. To calculate the constant of proportionality in the variance, one can calculate the expectation value of $\langle v^2\rangle$ for $t \to \infty$ to show that the classical equipartition $ m \langle \bb{v}^2 \rangle = k_BT$ only holds if the stochastic force and friction force are related by the relation
\begin{equation}
 \langle \xi(t) \xi(t')\rangle=2\gamma T \delta(t-t'). \label{eq:corr_ohmic} %\mathbf{I}_{3\times 3}
\end{equation}
This is the fluctuation-dissipation relation connecting the magnitude of fluctuations $\xi$ to the dissipation constant $\gamma$. The damping term of Eq. \eqref{eq:langevin_eq} is often referred to as Ohmic damping due to its correspondence with an Ohmic resistor in circuit theory \cite{weiss}.

In this example, the molecules of the solvent act as a thermal reservoir at temperature $T$. For any given initial velocity of the particle, the particle will drift toward thermal equilibrium with the reservoir and eventually achieve it. Building on this idea, Langevin forces are traditionally used in simulations to thermostat the system to a given temperature \cite{}. This allows one to either (i) simulate canonical ensemble at given temperature, (ii) push the system into thermal non-equilibrium by coupling atoms to Langevin thermostats at different temperatures, or (iii) to simulate dissipative and fluctuative processes driving the system to local equilibrium by Langevin thermostats at position-dependent temperatures.

\iffalse
\section{Langevin bath in simulations}

Langevin bath is typically used for three different tasks. In the first case, Langevin bath is used to simulate canonical ensemble (thermal equilibrium) by coupling all atoms to a bath at single temperaure $T$. In this case, the coupling constant $\gamma$ to the baths should typically be chosen small enough so that the coupling does not disturb the natural vibrational dynamics in the system. If the coupling is too small, however, the energy exchange with the bath is so slow that very long simulation runs are required to properly sample the available phase space.

In the second case, multiple baths at different temperatures are used to push the system into non-equilibrium. In this case, the baths act as heat sources and sinks, and the coupling constant $\gamma$ effectively determines the contact resistance with the reservoirs. While large $\gamma$ generally decreases the contact resistance to the reservoirs, it also increases the acoustic mismatch between thermalized and unthermalized atoms. Therefore, it should be carefully checked that the obtained results (such as thermal resistance) are not sensitive to the exact value of $\gamma$.

Finally, coupling to the Langevin bath can describe \textit{internal} processes driving the system into (local) thermal equilibrium. For example, the complicated phonon-phonon interactions giving rise to phonon creation and annihilation can be described in an effective manner by the fluctuating and dissipative Langevin forces, respectively. The resulting linearization of the equations of motion allows for solving the equations of motion directly in terms of the Green's function. To ensure current conservation, it is necessary to determine the bath temperatures self-consistently so that phonon creation and annihilation are balanced. This is the self-consistent heat bath model.
\fi
\section{General Langevin equation}

The original Langevin equation with the classical fluctuation-dissipation relation is typically used in molecular dynamics simulations due to its simplicity. In cases when the spectral properties of the coupling to the bath matter (for example to minimize the contact resistance between the bath and the system) or if quantum statistics must be accounted for, one must turn to the general Langevin theory \cite{weiss}.

In generaly Langevin theory, the reservoir is modeled as a collection of harmonic oscillators. The bath degrees of freedom are ''integrated out'' so that their interaction with the system under study is described effectively by the Langevin forces. In general, the friction and force then have temporal correlations and the Ohmic damping of Eq. \eqref{eq:langevin_eq} and Markovian force [Eq. \eqref{eq:corr_ohmic}] are replaced by more complicated expressions \cite{weiss}. The general Langevin equation reads \cite{dhar06}
\begin{equation}
 m\ddot{u}(t) =  F[u(t)] - \int_{0}^{\infty}dt' \Sigma(t') {u}(t-t') + \xi(t),
\end{equation}
where the auto-correlation function of the random force $\xi$ is related to the damping self-energy through the fluctuation-dissipation relation
\begin{equation}
 \langle \xi(t)\xi(t') \rangle = \int_{-\infty}^{\infty} \frac{d\omega}{2\pi} e^{-i\omega(t-t')} \hbar \Gamma(\omega) [f_B(\omega,T)+1].
\end{equation}
By Fourier transforming with respect to $t$ and $t'$ separately, Eq. (XXX) can be written in the form useful for calculations:
\begin{equation}
  \langle \xi(\omega)\xi(\omega') \rangle = 2\pi\hbar\delta(\omega+\omega') \Gamma(\omega) \left[f_B(\omega,T)+1 \right].
\end{equation}
Here $\Gamma(\omega)=-2\textrm{Im}[\Sigma(\omega)]$.

Typically, the integral term in the Langevin equation is written in terms of the velocity to identify it as a frictional force. To achieve this, we integrate partially in Eq. \eqref{}:
\begin{equation}
 m\ddot{u}(t) =  F[u(t)] - \int_{0}^{\infty}dt' M(t')\dot{u}(t-t') + \xi(t)
\end{equation}
The boundary terms in the integral are assumed to vanish  because we (i) define the integral function $M(t)=-\int_t^{\infty} dt' \Sigma(t')$ of $\Sigma(t)$ so that $M(t\to \infty)=0$ and (ii) the term proportional to $M(0)u(t)$ can be absorbed to the external force $F[u(t)]$ or eliminated by re-defining the displacements \cite{weiss}. 

Usually, the bath self-energy $\Sigma(\omega)$ is given to specify the coupling with the bath. Therefore, it is useful to derive an expression relating the bath self-energy to $M(t)$. This process is complicated by the fact that because $M(t)$ does not vanish at negative infinity, one cannot use the Fourier transform of $M(t)$ in the process. However, because only the values of $M(t)$ for $t>0$ play a role in Eq. \eqref{}, one can introduce a step-function in the integral and substitute the convenient definition $M^e(t)=\Theta(t)M(t)+\Theta(-t)M(-t)$:
\begin{equation}
 m\ddot{u}(t) =  F[u(t)] - \int_{-\infty}^{\infty}dt'\Theta(t') M^e(t')\dot{u}(t-t') + \xi(t).
\end{equation}
One can then easily show that the Fourier transform of $M^e(t)$ is related to the bath self-energy $\Sigma(\omega)$ through the coupling function $\Gamma(\omega)=-2\textrm{Im}[\Sigma(\omega)]$:
\begin{equation}
 \hat M^e(\omega) = \frac{\Gamma(\omega)}{\omega}.
\end{equation}
%The fluctuation-dissipation relation then becomes
%\begin{equation}
% \langle \xi(\omega)\xi(\omega') \rangle = 2\pi\hbar\delta(\omega+\omega') \omega M^e(\omega) \left[f_B(\omega,T)+1 \right].
%\end{equation}


\subsection{Ohmic damping}
In cases where the exact spectral properties of the bath do not matter, the simplest choice for the bath self-energy is
\begin{equation}
 \Sigma(\omega) = -i\gamma \omega \Theta(\omega_c-|\omega|),
\end{equation}
where $\omega_c$ is the cut-off frequency for the bath modes. Equation (XXX) then gives
\begin{equation}
 \hat M^e(\omega) = 2\gamma \Theta(\omega_c-|\omega|), 
\end{equation}
so the friction kernel $M^e(t)$ is 
\begin{equation}
 M^e(t) = 2 \gamma \delta_{\omega_c}(t),
\end{equation}
where 
\begin{equation}
 \delta_{\omega_c} (t) = \frac{1}{\pi} \frac{\sin \omega_c t}{t}.
\end{equation}
For $\omega_c\to \infty$, Eq. (XXX) tends to the Dirac Delta function and the friction term in the generalized Langevin equation reduces to the Ohmic damping in the classic Langevin equation (XXX).




%However, because the precise nature of the bath coupling should be irrelevant, Ohmic damping and memoryless force are typically assumed for simplicity in classical molecular dynamics simulations. The spectral density of the correlation functions reflects the frequency-dependence of the bath-system coupling and, for quantum systems, the Bose-Einstein statistics in the bath. 

