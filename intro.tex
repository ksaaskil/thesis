%\chapter{Introduction}

\section{Background}

\begin{itemize}
 \item Three primary carriers
\end{itemize}

\section{Energy transfer phenomena}

\begin{itemize}
 \item Interference, particles versus waves
 \item Boundaries and interfaces
 \item Ballistic versus diffusive, coherence
\end{itemize}


\section{Summary of experimental techniques}

\subsection{Thermal conductivity}

\begin{itemize}
 \item $3\omega$ technique
 \item Picosecond ultrasonic techniques (transient reflectance)
\end{itemize}

\subsection{Local temperature}

\subsection{Electromagnetic near-field transfer}

\subsection{Scanning thermal microscopy}

Measure the temperature of an AFM probe during the scan using either a thermocouple junction (measure voltage caused by temperature change) or microbolometer technique (measure change in resistance). In the latter, two leads are connected at the end of the probe by a Joule heating element which can be used either for temperature measurement by measuring its temperature change or for heating by driving current through it. In the constant power mode, the resistance of the heating element is measured by measuring the voltage in a Wheatstone bridge. If the voltage is fed back to the contact voltage, one can keep constant temperature at the resistor. 

Heat flow from the tip can be due to
\begin{itemize}
 \item Solid-solid conduction (this is what is wanted)
 \item Liquid-liquid conduction by the liquid meniscus between the tip and the sample, use UHV conditions
 \item Gas-gas conduction, use UHV conditions
 \item Near-field radiation between the tip and the sample
 \item Heat flow to cantilever
\end{itemize}

If the temperature of the tip, say, drops during the scan, this can be due to (a) lower local temperature, or (b) higher thermal conductivity at the sample spot. Also lower heat capacity is possible (?). 

Technique developed by Nonnenmacher (1992), Wickramasinghe (1992), Majumdar (1993), Pilkki (1994), etc.

Other methods to measure temperature are (see the review by Yue and Wang)
\begin{itemize}
 \item Optical methods based on the temperature-dependence of Raman or fluorescence signal of the measuring target (molecule, nanoparticle, etc.), which can also be used as the temperature sensor at the tip of an AFM, for example
 \item Near-field optical temperature measurement, with or without aperture
\end{itemize}

\subsection{Inelastic neutron scattering}
 \begin{itemize}
  \item Neutrons interact strongly only with the atomic nuclei (dipole scattering etc. typically weaker)
  \item Map the change in the neutron energy and momentum, one-phonon scattering processes sharply resolved among the multi-phonon process background
  \item Vary neutron energy, orientation of crystal and detection direction
  \item Gives phonon dispersion relations and broadenings, anharmonic effects mapped recently e.g. in doi:10.1038/nmat3035 and doi:10.1038/nnano.2013.95
 \end{itemize}

