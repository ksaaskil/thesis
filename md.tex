\section{Molecular dynamics method}

Molecular dynamics simulations are useful for modeling energy transfer by lattice vibrations. In MD, the classical Newton's equations of motion 
\begin{equation}
 m \frac{d^2 \bb{x}}{dt^2} = \bb{F}_i 
\end{equation}
for each atom $i$ is integrated numerically. Here the force $\bb{F}_i$ is derived from the interatomic potential $U$ as $\bb{F}=-\partial U/\partial \bb{r}_i$. Values of macroscopic observables such as heat current are extracted by time-averaging from the simulation. Based on the ergodicity principle, the time averages are expected to equal the statistical average over the corresponding ensemble. 

% One key aspect of the method is the assumption of ergodicity: statistical ensemble average can be calculated from a time average, meaning that  %In a microcanonical simulation, for example, the system should sample all the states that lie on the constant-energy manifold determined by the initial conditions. 

In the following, we briefly go through the most important aspects related to a MD simulation.

%\begin{itemize}
% \item History in heat transfer (FPU,...)
% \item Integrator (time-reversibility, symplecticity, okayness of wrong trajectories)
% \item Equilibrium and non-equilibrium simulations + wave packet dynamics
% \item Parallelization, domain decomposition
% \item Interatomic potentials (how to calculate, FPU, Brenner, Tersoff)
% \item Thermalization (Langevin, Nose-Hoover, energy exchange, Maxwell)
%\end{itemize}

\subsection{Time integration}
%As stated above, the continuous-time equations of motion are integrated numerically using a finite-difference method. The pool of the finite-difference methods includes, e.g., Euler methods, Verlet methods, Runge-Kutta methods, the leap-frog method and predictor-corrector methods. Before discussing the properties of a good integrator, it is necessary to understand that none of the integrators can follow the ''true'' trajectory of a many-particle system with thousands of particles for very long times. Due to the so-called Lyapunov instability related to the chaoticity in a many-particle systems, any two many-particle trajectories with minimal differences in the initial conditions diverge from each other exponentially fast. Because of truncation and round-off errors present in any simulation, one therefore cannot even hope to solve the exact particle trajectories. Fortunately, solving the exact trajectories is not even necessary: it is generally believed that the numerical trajectories (''shadow orbits'') are, in some sense, representative of the true trajectories and can therefore be used for collecting statistics.

%With this remark in mind, the time-integration is anyway expected to satisfy a few requirements. First, the method should be reasonably accurate even at long time steps to ensure that large systems can be simulated efficiently. Second, the method should be time-reversible, meaning that if the velocities were reversed at any time-step, the system would ''rewind'' along its earlier trajectory. The requirement of time-reversibility is closely related to the third requirement: energy conservation. Because the energy conservation is a fundamental property of Newton's equations of motion, the simulation should present as little short-term and long-term energy drift as possible. While some non-time-reversible algorithms can have very small short-term energy drift due their high accuracy, they often exhibit large long-term energy drift. The long-term drift arises from the non-symplecticity of these integrators: the size of the phase space volume element is not conserved in the integration.

As stated above, the Newton's equations of motion 
\begin{equation}
 m \ddot{\bb{x}}(t) = \bb{F}[\bb{x}(t)]
\end{equation}
for the position vector $\bb{x}(t)$ and force $\bb{F}$ are integrated numerically using a finite-difference method. The pool of the finite-difference methods includes, e.g., Euler methods, Verlet methods, Runge-Kutta methods, the leap-frog method and predictor-corrector methods \cite{}. In all the simulations of this work, the velocity Verlet algorithm was employed to integrate the equations of motion. While velocity Verlet exhibits moderate short-term energy drift, it is very simple to implement, efficient and it possesses very small long-term energy drift due to its being both time-reversible and symplectic \cite{}. The velocity Verlet equations for the position vector $\bb{x}$ and velocity $\bb{v}=\dot{\bb{x}}$ are
\begin{alignat}{2}
  \bb{x}(t+\Delta t) &= \bb{x}(t) + \bb{v}(t)\Delta t+  \frac{1}{2}\bb{a}(t) \Delta t^2 + \mathrm{O}(\Delta t^4) \\
  \bb{v}(t+\Delta t) &= \bb{v}(t) + \frac{ \bb{a}(t)+\bb{a}(t+\Delta t)}{2} \Delta t+ \mathrm{O}(\Delta t^2) .
\end{alignat}

% \textit{Lyapunov instability}



\subsection{Nonequilibrium simulation}

\subsection{Observables and the spectral heat current formula}

%\subsection{
