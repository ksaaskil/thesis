\chapter{Theory}
\label{chap:theory}
%\begin{quote}
% \textit{This chapter discusses the theory of energy transfer.}
%\end{quote}


In the publications included in this thesis, we have employed molecular dynamics (MD) simulations and Green's function (GF) calculations to model vibrational heat transfer in nanostructures. It is common to both of these methods that one needs to choose how to describe interatomic interactions and the coupling to external heat baths. Before describing MD and GF calculations in Secs. \ref{sec:md} and \ref{sec:gf}, we therefore briefly review the chosen interatomic potentials and the theory of Langevin heat baths below in Secs. \ref{sec:th_interatomicpotential} and \ref{sec:th_langevin}. When discussing the MD method, we also present the recently developed expression for the spectral heat current distribution.

For electromagnetic energy transfer, we have employed the same Langevin theory as in vibrational heat transfer to describe the microscopic dipolar thermal fluctuations. This theory is presented in Sec. \ref{sec:em_methods}

\section{Vibrational heat transfer}

\subsection{Lattice Hamiltonian and phonons}
\label{sec:theory_vibtheory}
%As discussed in Chap. 1, propagating lattice vibrations carry heat in crystalline solids, and the quanta of such propagating vibrations are called phonons. The theoretical description is based essentially on the equations of motion for the atoms in the solid. Because fully quantum description does not easily allow for accounting for non-linear dynamics, which play an essential role in any system with non-negligible phonon-phonon interactions, the discussion below treats the atomic dynamics classically. 

%\begin{itemize}
% \item The goals of this work
%\end{itemize}
The theoretical description of lattice heat transfer is based on the dynamical equations of motion for the atoms constituting the lattice. The equations of motion are generally dictated by the Hamiltonian \cite{ziman}
\begin{equation}
 \ca{H} = \sum_{i=1}^N \frac{\bb{p}_i^2}{2m_i} + \ca{V}(\bb{r}_1,\dots,\bb{r}_N). \label{eq:th_hamiltonian}
\end{equation}
Here $\br_i$, $\bp_i$, and $m_i$ are the position, momentum and mass of atom $i$, respectively. The total number of atoms (which can also be infinite) is denoted by $N$. The first term of Eq. \eqref{eq:th_hamiltonian} is the total kinetic energy of the atoms and the second term $\ca{V}$ is the interatomic potential energy responsible for the interatomic interactions. The choice of the potential energy function $\ca{V}$ is crucial for an accurate description of the lattice dynamics and, consequently, of energy transfer. Models for $\ca{V}$ used in this work are explained in detail bwlow in Subsection \ref{sec:th_interatomicpotential}.

Applying Hamilton's equations of motion $\dot{\br}_i=\partial H/\partial \bp_i$ and $\dot{\bp}_i=-\partial \ca{H}/\partial \br_i$ \cite{fetter} gives Newton's law
\begin{equation}
 m_i \ddot{\br}_i = \bb{F}_i, \label{eq:th_eom}
\end{equation}
where the force acting on atom $i$ is
\begin{equation}
 \bb{F}_i = - \frac{\partial \ca{V}}{\partial \bb{r}_i}. \label{eq:th_force}
\end{equation}
For given initial conditions $\br_i(0)$ and $\dot{\br}_i(0)$, Eq. \eqref{eq:th_eom} determines the time evolution of atomic trajectories. To model energy transfer, the equations of motion are supplemented by terms accounting for coupling to external heat baths. In this work, we mostly employ Langevin heat baths that turn the equations of motion into stochastic equations and ensure that the long-term atomic trajectories correctly sample the non-equilibrium statistical ensemble. Langevin theory is presented in Sec. \ref{sec:th_langevin}.

Equation \eqref{eq:th_eom} generally describes the motions of atoms and molecules in solid, gas, and liquid systems. In solids, the atoms vibrate close to their equilibrium positions $\br_i^0$ and one can gain more insight into the lattice dynamics by only considering small displacements from the equilibrium. The positions $\br_i^0$ are defined by the condition of zero force:
\begin{equation}
 \left. \frac{\partial \ca{V}}{\partial \br_i} \right|_{\br_j=\br_j^0 \quad \forall j} = 0. \label{eq:th_zeroforce}
\end{equation}
Assuming that the atoms remain close to the equilibrium positions, one can expand the potential energy in Taylor series in terms of the displacements $\bu_i=\br_i-\br_i^0$:
\begin{equation}
 \ca{V} = \ca{V}_0 + \frac{1}{2} \sum_{i,j} \sum_{\alpha,\beta} u_i^{\alpha} K_{ij}^{\alpha \beta} u_j^{\beta}  + \ca{O}(u^3). \label{eq:th_V_taylor}
\end{equation}
Here the Cartesian coordinate directions $\alpha,\beta \in \{x,y,z\}$ have been written explicitly for clarity and the second-order term is proportional to the ''force constant''
\begin{equation}
 K_{ij}^{\alpha\beta} = \left. \frac{\partial^2 \ca{V}}{\partial u_i^{\alpha} \partial u_j^{\beta}} \right|_{\bu=\mathbf{0}}. \label{eq:th_K_def}
\end{equation}
The first-order derivative term in \eqref{eq:th_V_taylor} vanished based on Eq. \eqref{eq:th_zeroforce} and the last term is of third order in displacements.

In the case that the third-order term can be neglected, employing Eq. \eqref{eq:th_V_taylor} in the equation of motion \eqref{eq:th_eom} gives the system of linear equations
\begin{equation}
 m_i \ddot{u}_i^{\alpha} = - \sum_j \sum_{\beta} K_{ij}^{\alpha\beta} u_j^{\beta}.
\end{equation}
Following standard eigenmode theory \cite{fetter}, the eigenmodes of the system can be found by diagonalizing the matrix $D_{ij}^{\alpha\beta} = (m_i\omega^2 \delta_{ij}\delta
^{\alpha\beta}-K_{ij}^{\alpha\beta})$. In a periodically repeating crystal, the eigenmodes can be labeled by the wavevectors $\bb{q}$ belonging to the first Brillouin zone \cite{ziman} and the branch $p \in \{1,\dots,3N_{\textrm{cell}}\}$, where $N_{\textrm{cell}}$ is the number of atoms in the unit cell. The eigenmodes are called phonon modes, while phonons are the discrete quanta of eigenmode occupation. The eigenfrequencies $\omega(\bb{q},p)$ form the phonon bandstructure, specifying the relation between the wavevectors and frequencies supporting propagating phonon modes. As an example, Fig. \ref{fig:th_nika} shows the phonon bandstructure of graphene, a single monolayer of graphite.

\begin{figure}
\begin{center}
 \includegraphics[width=8.6cm]{pics/nika09_fig3.pdf}
 \caption{Phonon bandstructure of graphene, calculated using the valence force field method \cite{nika09}. The two-dimensional bandstructure is plotted along one-dimensional lines between special points in graphene reciprocal lattice, denoted by $\Gamma$, $M$ and $K$. Because graphene has two atoms per unit cell, there are altogether six phonon branches. The three branches that have vanishing frequencies at the $\Gamma$ point are called longitudinal acoustic (LA), transverse acoustic (TA) and out-of-plane acoustic (ZA). The optical modes LO, TO and ZO are labeled similarly. Reprinted with permission from Ref. \cite{nika09}.}
\label{fig:th_nika}
\end{center}
\end{figure} 

When the anharmonic part in Eq. \eqref{eq:th_V_taylor} is neglected, the phonon eigenmodes are exact eigenmodes of the system and cannot dissipate their energy, giving rise to infinite thermal conductivity \cite{ziman}. The anharmonic terms give rise to phonon-phonon scattering \cite{ziman}, which is the primary phonon decay mechanism in crystalline solids at high temperatures. In Publications \cp{fpu}, \cp{fpu2}, \cp{spectral}, \cp{cnt}, and \cp{twinning}, we have employed classical molecular dynamics simulations fully accounting for anharmonic scattering. In \citepub{gf}, anharmonic effects are mimicked by the self-consistent heat bath model \cite{bolsterli70}, allowing for the inclusion of quantum statistics as well. These methods are explained in more detail below in Chap. XXX.

\subsection{Models for interatomic potential energy}
\label{sec:th_interatomicpotential}

%Typically, the analytical form of the interatomic potential is inferred from quantum-mechanical calculation and the free parameters are fitted to reproduce experimentally known quantities such as the lattice constant, bulk modulus, atomization energy, and so on. For this reason, the interatomic potentials are often called semi-empirical. In chemistry, the term force field is used instead of interatomic potential.

As mentioned in Sec. \ref{sec:intro_vibtheory}, a crucial physical aspect of correctly describing the lattice dynamics and, therefore, vibrational energy transfer is the choice of interatomic potential energy function $\ca{V}$. In general, the interatomic potential consists of pair potential terms and many-body terms. 
%, which we assume to only consist of the three-body terms $V^{(3)}(\bb{r}_i,\bb{r}_j,\bb{r}_k)$:
%\begin{equation}
% U(\bb{x}) = \frac{1}{2}  \sum_{ i,j } V^{(2)}(|\bb{r}_{i}-\bb{r}_j|) + \frac{1}{6} \sum_{i,j,k} V^{(3)}(\bb{r}_i,\bb{r}_j,\bb{r}_k)
%\end{equation}
A very simple example of a pure pair-potential is the Fermi-Pasta-Ulam (FPU) potential used by Fermi, Pasta, and Ulam to investigate the minimal necessary conditions for thermalization in one-dimensional system. In the FPU model, atoms with displacement $u_i$ from the equilibrium position are assumed to be connected to their nearest neighbors by anharmonic springs with the pair-wise energy of the form
\begin{equation}
  V_{ij}^{\textrm{FPU}} = \frac{1}{2} k (u_i-u_j)^2 + \frac{\alpha}{3} (u_i-u_j)^3+ \frac{\beta}{4} (u_i-u_j)^4, \label{eq:th_fpu}
\end{equation}
which is referred to as the FPU potential. The FPU potential can be considered to arise from the Taylor expansion of a more realistic potential (such as LJ). The models for $\beta=0$ and $\alpha=0$ are known as $\alpha$-FPU and $\beta$-FPU, respectively, and both have been employed extensively in investigating thermalization and thermal conduction in low-dimensional systems \cite{}. In Publications \cp{fpu} and \cp{fpu2}, the $\beta$-FPU potential was used to model the anharmonic interactions in a square lattice. In Publication \cp{gf}, the anharmonic interactions were mimicked by the coupling to self-consistent heat baths (see Sec. \ref{sec:gf}), and only the harmonic term of Eq. \eqref{eq:th_fpu} was included ($\alpha=\beta=0$).

Another very common pair-potential is the Lennard-Jones (LJ) potential \cite{allentildesley}
\begin{equation}
 V_{ij}^{\textrm{LJ}}(r_{ij}) = 4\varepsilon \left[\left( \frac{\sigma}{r_{ij}}\right)^{12}-\left( \frac{\sigma}{r_{ij}}\right)^6  \right],
\end{equation}
where $\varepsilon$ is the interaction energy, $\sigma$ determines the equilibrium distance $r_0$ of atoms ($r_0=2^{1/6}\sigma$ for two particles), and $r_{ij}$ is the interparticle distance. The repulsive term $(\sigma/r_{ij})^{12}$ models the strong atomic repulsion at short distances, arising from the overlapping of electron clouds. The attractive term $-(\sigma/r_{ij})^{6}$ accounts for the weak van der Waals attraction at large distances, arising from the interaction of the fluctuating dipole moments due to, e.g., electron polarization. The LJ potential accurately describes interatomic interactions between noble gas atoms such as argon and, thanks to its simple form, it is also often used to investigate the qualitative features of heat transfer in solids \cite{}. The LJ potential was used in Publication \cp{spectral} to model the interatomic interactions in investigating the spectral conductance between mass-mismatched solids arranged in a face-centered cubic lattice. %Because the LJ interaction does not account for the local environment, however, the LJ potential cannot describe, e.g., covalent bonding. %It is used a constituent in more complicated potentials to describe the van der Waals attractions. Due to its simple form, it is also often used 


Pure pair-potentials such as the Lennard-Jones potential cannot describe, e.g., covalent bonding, where the strength of local bonding is strongly influenced by the environment. Therefore, a more sophisticated potential is needed to model, for example, carbon materials. A typical example of a many-body potential is the Tersoff potential \cite{tersoff88b}
\begin{equation}
 V_{ij}(r_{ij}) = f_C(r_{ij}) \left[A e^{-\lambda_1 r_{ij}} - B b_{ij} e^{-\lambda_2 r_{ij}}) \right],
\end{equation}
where the taper function 
\begin{equation}
 f_C ( r) = \left\{ \begin{array}{ll}
                     1 & \textrm{for } r<R-D,\\
		     \frac{1}{2}-\frac{1}{2}\sin\left(\frac{\pi}{2}\frac{r-R}{D} \right) & \textrm{for } R-D < r < R+D, \\
		     0 & \textrm{for } r>R+D
                    \end{array}
 \right.
\end{equation}
gradually turns off the pair-wise interaction between $r_{ij}=R-D$ and $r_{ij}=R+D$. The strength of interatomic attraction is controlled by the coefficient
\begin{equation}
 b_{ij} = \left( 1+\beta^n \zeta_{ij}^n \right)^{-1/(2n)},
\end{equation}
where the dependence on the local environment appears in the definition 
\begin{equation}
 \zeta_{ij} = \sum_{k\neq i,j} f_C(r_{ik}) g(\Theta_{ijk}) \exp\left[\lambda_3^3(r_{ij}-r_{ik})^3 \right] .
\end{equation}
Here $\Theta_{ijk}$ is the angle between bonds $ij$ and $ik$. The angle function is defined as 
\begin{equation}
 g(\Theta) = 1 + \frac{c^2}{d^2} - \frac{c^2}{d^2 + (\cos \Theta-\cos \Theta_0)^2}
\end{equation}
The parameters $R$, $D$, $A$, $\lambda_1$, $B$, $\lambda_2$, $\beta$, $n$, $\lambda_3$, $c$, $d$, and $\Theta_0$ depend on the material under study. The Tersoff parameters for carbon systems were originally fit to the experimentally known cohesive energies of various carbon systems and the lattice constant and bulk modulus of diamond \cite{tersoff88a}. Recently, Lindsay and Broido \cite{lindsay10} suggested an improved set of parameters found by giving more weight to matching the experimentally measured phonon dispersion for graphite. This optimized Tersoff potential was used in Publication \cp{cnt} for carbon nanotubes. In \citepub{twinning}, many-body Stillinger-Weber potential \cite{stillinger85} was used to model interactions between Si atoms constituting Si nanowire. 


\subsection{Langevin theory} 
\label{sec:th_langevin}
The equations of motion \eqref{eq:th_eom} only describe the interactions between the constituents of the system under study. To enable steady-state energy transfer, some of the degrees of freedom must be coupled to external heat baths acting as heat sources and sinks. Because Langevin heat baths are employed in all but one publication included in this thesis, we briefly review the Langevin theory.

In Langevin theory, the particle coupled to the bath is imagined to interact with a collection of harmonic oscillators at a prescribed temperature. The bath degrees of freedom are ''integrated out'' so that their interaction with the system under study is described effectively by the Langevin forces \cite{weiss}. The general Langevin equation obtained through such a procedure reads \cite{dhar06}
\begin{equation}
 m\ddot{\bu}_i(t) =  \bb{F}_i(t) - \int_{0}^{\infty}dt' \Sigma_i(t') \bb{u}_i(t-t') + \xi_i(t), \label{eq:th_eom_langevin}
\end{equation}
where, for the simplicity of discussion, we assume that the baths are spatially uncorrelated so that the bath self-energy $\Sigma_i$ is spatially local. The self-energy describes the dynamical interactions and damping induced by the coupling to the heat bath oscillators. The force $\bb{F}_i$ is due to interactions with particles not in the reservoir and the auto-correlation function of the random force $\xi_i$ is related to the damping self-energy $\Sigma_i(t)$ and bath temperature $T_i$ through the fluctuation-dissipation theorem (FDT) \cite{dhar06}
\begin{equation}
 \langle \xi_i(t)\xi_i(t') ^T\rangle = \int_{-\infty}^{\infty} \frac{d\omega}{2\pi} e^{-i\omega(t-t')} \hbar \Gamma_i(\omega) \left[f_B(\omega,T_i)+\frac{1}{2}\right] \bb{I}_{3\times3}. \label{eq:th_xixit}
\end{equation}
Here $\Gamma_i(\omega)=-2\textrm{Im}[\Sigma_i(\omega)]$ is called the bath coupling function \cite{dhar06}. The quantum statistics appear through the Bose-Einstein occupation function $f_B(\omega,T)=\left\{\exp[\hbar\omega/(k_BT)]-1 \right\}^{-1}$. By Fourier transforming with respect to $t$ and $t'$ separately, Eq. \eqref{eq:th_xixit} can be written in the form useful for calculations:
\begin{equation}
  \langle\tilde  \xi_i(\omega)\tilde \xi_i(\omega')^T \rangle = 2\pi\hbar\delta(\omega+\omega') \Gamma_i(\omega) \left[f_B(\omega,T_i)+\frac{1}{2} \right] \bb{I}_{3\times3}. \label{eq:th_xixiom}
\end{equation}

To simulate Langevin dynamics in a MD simulation, it is useful to write the integral term appearing in the Langevin equation in terms of the velocity $\dot{u}(t)$. To achieve this, one can integrate in Eq. \eqref{eq:th_eom_langevin} partially to get
\begin{equation}
 m\ddot{\bu}_i(t) =  \bb{F}_i(t) - \int_{0}^{\infty}dt' M(t')\dot{\bb{u}}_i(t-t') + \xi_i(t). \label{eq:th_langevin_Mt}
\end{equation}
The boundary terms appearing in the partial integration are assumed to vanish because we (i) define the integral function $M(t)=-\int_t^{\infty} dt' \Sigma(t')$ of $\Sigma(t)$ for positive $t$ so that $M(t\to \infty)=0$ and (ii) the term proportional to $M(0)u(t)$ can be absorbed to the external force $F[u(t)]$ or eliminated by re-defining the displacements \cite{weiss}. The classical Langevin equation is obtained by choosing a very rapidly decaying $M(t)$ and taking the limit of vanishing decay time, allowing for arriving at the classic Langevin equation \cite{zwanzig}
\begin{equation}
 m\ddot{\bu}_i(t) =  \bb{F}_i(t) - m\gamma \dot{\bu}_i(t) + \xi_i(t). \label{eq:th_ohmic}
\end{equation}
This form of damping, proportional to the instantaneous velocity $\dot{\bu}(t)$ and the friction constant $\gamma$, is called Ohmic damping due to its analogue with a resistor in an electrical circuit \cite{weiss}. This form can be shown to give rise to a frequency-independent phonon relaxation time $\tau=1/\gamma$ \cite{li09jap}. The corresponding FDT \eqref{eq:th_xixiom} for the force variance is, for Ohmic damping,
\begin{equation}
 \langle \tilde \xi_i(\omega) \tilde \xi_i(\omega')^T \rangle = 4\pi \delta(\omega+\omega') \hbar \omega \gamma \left[f_B(\omega,T_i)+ \frac{1}{2} \right] \bb{I}_{3\times 3}. \label{eq:th_xixiom_ohmic_qm}
\end{equation}
In the classical high-temperature limit relevant for classical molecular dynamics, one gets the classical FDT \cite{zwanzig}
\begin{equation}
 \langle \xi_i(t) \xi_i(t')^T\rangle=2\gamma k_B T_i \delta(t-t') \bb{I}_{3\times 3}. \label{eq:th_corr_ohmic} 
\end{equation}

In this thesis, we employ the Ohmic damping of Eq. \eqref{eq:th_ohmic} due to its simplicity. In Publications \cp{fpu}, \cp{fpu2}, \cp{spectral}, and \cp{cnt}, Ohmic Langevin heat baths are used as hot and cold heat baths in the molecular dynamics simulation. In accordance with the classical dynamics, the classical FDT \eqref{eq:th_corr_ohmic} is employed for force variance. \citepub{twinning} employs the classical Nose-Hoover heat baths \cite{hoover85} instead of Langevin heat baths. In Publication \cp{gf}, Langevin heat baths act not only as external thermal reservoirs but also as pathways for phonon creation and annihilation inside the system under study. In Publication \cp{dipole}, Langevin baths are used to model thermal fluctuations and dissipation of dipole moments. Because the equations of motion are linear in the two latter cases, we can also account for quantum statistics by using the quantum fluctuation-dissipation theorem \eqref{eq:th_xixiom_ohmic_qm}.


\section{Electromagnetic energy transfer}

% \subsection{Theoretical background}

\label{sec:theory_emtheory}

\subsection{Field due to an oscillating dipole}

The theoretical description of electromagnetic energy transfer between oscillating dipoles is based on Maxwell equations \cite{novotny}. In the non-magnetic materials with no free charges that are considered in this work, the electromagnetic fields arise from the fluctuating electric polarization fields inside the bodies, and the Maxwell equations for the electric field $\bE(\br,t)$ and magnetic field $\bb{H}(\br,t)$ read \cite{novotny}
\begin{subequations}
\begin{align}
  \nabla \times \bE(\br,t) &= - \mu_0 \frac{\partial \bb{H}(\br,t)}{\partial t}, \label{eq:th_maxwell1} \\
  \nabla \times \bb{H}(\br,t) &= \varepsilon_0 \frac{\partial \bb{E}(\br,t)}{\partial t} + \frac{\partial \bb{P}(\br,t)}{\partial t}, \label{eq:th_maxwell2} \\
   \nabla \cdot \bb{H}(\br,t) &= 0, \\
   \nabla \cdot \bb{E}(\br,t) &= 0.
\end{align}
\end{subequations}
Equation \eqref{eq:th_maxwell2} shows that a temporal change in the polarization density $\bb{P}(\br,t)$ gives rise to a magnetic field, which in turn induces an electric field according to Eq. \eqref{eq:th_maxwell1}. The induced electromagnetic field carries energy flux, whose magnitude and direction are given by the Poynting vector $\bb{S}(\br,t)=\bb{E}(\br,t)\times \bb{H}(\br,t)$ \cite{novotny}.

To determine the amount of energy radiated by a fluctuating dipole, one needs to solve for the electric and magnetic fields emitted by the dipole current density distribution $\bb{j}(\br',t)=\partial \bb{P}(\br,t)/\partial t$. As shown in detail in Ref. \cite{novotny}, the electric field is given in frequency-domain by 
\begin{equation}
 \tilde \bE(\br,\omega) = \tilde \bE_0(\br,\omega) + i \omega \mu_0 \int_V d\mathbf{r}' \mathbb{G}(\br,\br';\omega) \tilde{\bb{j}}(\br',\omega). \label{eq:th_Etilde}
\end{equation}
Here $\tilde{\bE}_0(\br,\omega)$ is the electric field arising from sources other than the oscillating dipoles, volume $V$ encloses the dipoles and $\mathbb{G}(\br,\br';\omega)$ is the electromagnetic Green's dyadic found by solving the Helmholtz equation \cite{novotny}
 \begin{equation}
 \nabla \times \nabla \times \gem(\bb{r},\br';\omega) - (\omega^2/c^2) \epsenv(\br,\omega)\gem(\bb{r},\br';\omega)  =  \delta(\bb{r}-\br')\unitdyadic. \label{eq:intro_gemdef}
\end{equation}
Here $c$ is the speed of light and $\epsenv(\br,\omega)$ is the relative dielectric constant of the environment. The Green's dyadic can generally be decomposed into the free-space and scattered parts as
\begin{equation}
 \mathbb{G}(\br,\br';\omega ) = \mathbb{G}_0(\br,\br';\omega ) + \mathbb{G}_s(\br,\br';\omega ). \label{eq:th_G_decomp}
\end{equation}
The first term, which corresponds to the field radiated by the dipole in absence of any scattering events, is \cite{novotny}
\begin{equation}
 \mathbb{G}_0(\br,\br';\omega) = \left[\mathbf{I}_{3\times 3} + \frac{1}{k_0^2} \nabla \nabla \right] \frac{e^{ik_0|\br-\br'|}}{4\pi|\br-\br'|}.
\end{equation}
Here $k_0=\omega/c$ is the wavevector in vacuum and $\mathbf{I}_{3\times 3}$ is the $3\times 3$ identity matrix. The second term $\mathbb{G}_s(\br,\br';\omega)$ accounts for the scattering of the emitted field by the inhomogeneities in the environment such as reflecting walls. The decomposition \eqref{eq:th_G_decomp} is useful, because the two terms behave differently for $\br\to \br'$: the dyadic $\mathbb{G}_0$ diverges for $\br\to \br'$, but the scattering part $\mathbb{G}_s$ is smooth \cite{novotny}. Having the expression \eqref{eq:th_Etilde} for the electric field, the magnetic field $\tilde{\bb{H}}(\br,\omega)$ can be solved from the first Maxwell equation \eqref{eq:th_maxwell1}, and one can calculate the Poynting vector $\bb{S}$.

\subsection{Fluctuational electrodynamics}

To calculate the energy transfer between bodies, we need an equation specifying the relation between the fluctuations in dipole moments and the material's optical properties and temperature. The traditional approach is the fluctuational electrodynamics (FED) theory pioneered by Rytov \cite{rytov} and Lifshitz \cite{lifshitz55}. The core of FED is the fluctuation-dissipation relation \cite{novotny,agarwal75_1}
\begin{equation}
 \langle \tilde{\bp}(\omega)\tilde{\bp}(\omega')^T\rangle = 4\pi \hbar \delta(\omega+\omega')  \textrm{Im}[\alpha(\omega)] \left[f_B(\omega,T)+\frac{1}{2} \right], \label{eq:th_fed_fdt}
\end{equation}
which is used to relate the stochastic fluctuations in the local dipole moment $\tilde{\bp}(\omega)$ (which is the local dipole density integrated over a small volume) to the imaginary part of the dipole polarizability dyadic $\alpha(\omega)$ and dipole temperature $T$. 

While Eq. \eqref{eq:th_fed_fdt} has been used to successfully calculate energy transfer rates in various situations \cite{}, there are two arguments supporting a more microscopic approach. First, because FED relies on an effective medium property, the local polarizability, applying the theory to very small systems requires great care. It was noted only recently by Manjavacas and Abajo de Carc\'ia \cite{manjavacas12} that the fluctuation-dissipation relation connecting the polarization to the polarizability must be modified when local radiative corrections become important to ensure that non-absorbing particles do not emit thermal radiation. Starting from a more microscopic theory would make it possible to avoid resorting to effective medium parameters in the formulation. Second, one can envision  when the optical phonons responsible for electromagnetic radiation cannot be considered to be decoupled from the acoustic phonons responsible for ''phonon radiation''. In such cases, it is necessary to describe the full lattice dynamics and its coupling to the electromagnetic field microscopically. 

In Publication \cp{dipole}, we developed such a microscopic generalization of fluctuational electrodynamics, basing the description of thermal fluctuations on writing quantum Langevin equations for the microscopic dipole oscillations. By starting from the microscopic equations of motion, we could straightforwardly derive expressions for heat transfer rates between dipoles in an inhomogeneous environment in full analogy to the phononic case treated in \citepub{gf}, directly accounting also for local radiative corrections. The theory is presented in Sec. \ref{sec:em_methods}.


\section{Electron transport}

\subsection{Tight-binding model}




% The thermal background field $\Eenvhat$ has zero average $\langle \Eenvhat \rangle=0$ and its symmetrized autocorrelation function satisfies the fluctuation-dissipation relation 


% \subsection{Introduction to Green's functions}
% 
% \label{sec:gf_linear}
% Green's function method is based on inverting the ''equation of motion operator'', which we will discuss later. For a general non-homogenous equation of the form
% \begin{equation}
%  \mathcal{L} f = g,
% \end{equation}
% where $\mathcal{L}$ is a linear operator and $g$ is the source function, symbolic solution in terms of the Green's function $\mathcal{G}$ is
% \begin{equation}
%  f = \mathcal{G} g.
% \end{equation}
% The Green's function $\mathcal{G}$ is defined as the inverse of $\mathcal{L}$:
% \begin{equation}
%  \mathcal{L} \mathcal{G} = I,
% \end{equation}
% where $I$ is the identity operator. Since $\mathcal{L}$ is linear, solution for 
% \begin{equation}
%  \mathcal{L} f = g_1 + g_2
% \end{equation}
% is the sum of solutions
% \begin{equation}
%  f = \mathcal{G}g_1 + \mathcal{G} g_2.
% \end{equation}
% Calculating $\mathcal{G}$ for a given $\mathcal{L}$ determines, therefore, the solution for any source function $g$. 
% 
% 
% 
% \subsection{Quantum mechanical Green's functions}
% 
% For completeness, we also briefly discuss the Green's functions that appear in the quantum-mechanical many-body problem. These functions are directly defined as statistical averages of different correlation functions and, at first sight, bear no resemblance to the Green's function discussed in Sec. \ref{sec:gf_linear}. The most used two-particle Green's functions are \cite{wang08}
%  \begin{alignat}{2}
%    G^R(t,t') &= -i\theta(t-t') \langle [\bb{u}(t), \bb{u}(t')^T] \rangle \\
%    G^A(t,t') &= i\theta(t'-t) \langle [\bb{u}(t), \bb{u}(t')^T] \rangle\\
%    G^>(t,t') &= -i\langle \bb{u}(t) \bb{u}(t')^T \rangle\\
%    G^<(t,t') &= -i\langle \bb{u}(t') \bb{u}(t)^T \rangle^T	 \\
%    G^t(t,t') &= \theta(t-t') G^>(t,t') + \theta(t'-t) G^<(t,t') \\
%    G^{\bar t}(t,t') &=\theta(t'-t) G^>(t,t') + \theta(t-t') G^<(t,t')  ,
%  \end{alignat}
% which are called the retarded, advanced, greater, lesser, time-ordered and anti-time-ordered Green's functions, respectively. The operators appearing inside the expectation values are written in Heisenberg picture. Out of the six Green's functions, only three are linearly independent and, in steady-state, the number of independent functions is reduced to two. In equilibrium, one of the Green's functions determines the others, and typically $G^R$ is considered. Note that $G^R$ satisfies
% \begin{alignat}{2}
%  \partial_t G^R(t,t')  &= -i \delta(t-t')  \langle [\bb{u}(t), \bb{u}(t')^T] \rangle -i \theta(t-t') \langle [\dot{\bb{u}}(t),\bb{u}(t')^T ] \rangle \\
%   &= -i \theta(t-t') \langle [\bb{p}(t),\bb{u}(t') ]^T \rangle
% \end{alignat}
% and
% \begin{alignat}{2}
%  \partial_t^2 G^R(t,t') &= - i \delta(t-t') \langle [\bb{p}(t),\bb{u}(t') ]^T \rangle - i \theta(t-t') \langle [\dot{\bb{p}}(t),\bb{u}(t')^T] \rangle \\
%   &= - \delta(t-t')\bb{I}  - i \theta(t-t') \langle [\dot{\bb{p}}(t),\bb{u}(t')^T] \rangle .
% \end{alignat}
% For a quadratic Hamiltonian 
% \begin{equation}
%  \mathcal{H} = \frac{\bb{p}^2}{2} + \frac{1}{2} \bb{u}^T \bb{K} \bb{u},
% \end{equation}
% the Heisenberg equation of motion for $\bb{p}(t)$ is 
% \begin{equation}
%  \dot{\bb{p}}(t) = - \bb{K} \bb{u}(t),
%  \label{eq:dotpt}
% \end{equation}
% so 
% \begin{equation}
%  \partial_t^2 G^R_{ij} (t,t') = - \delta(t-t') \delta_{ij} - K_{ik} G^R_{kj}(t,t').
% \end{equation}
% Fourier transformation then gives the familiar Green's function
% \begin{equation}
%  G^R(\omega) = [(\omega+i\eta)^2-\bb{K}]^{-1}
% \end{equation}
% from the last section. This short calculation justifies the name Green's function. Note that for an interacting system, Eq. \eqref{eq:dotpt} would not be valid and the hiearchy of equations of motion would not close.
% 
% The usefulness of Green's functions in the statistical mechanics of quantum-mechanical systems lies in the facts that (1) they can be used to calculate all thermodynamic observables \cite{negele}, and (2) they allow an easy and intuitive perturbative expansion that can be represented as Feynman diagrams \cite{negele,fetter2}. At zero and non-zero temperature, the diagrammatic expansion in terms of the interaction parameter is carried out for the time-ordered Green's function and the Matsubara Green's function, respectively. Methods such as functional renormalization group \cite{metzner12,saaskilahti11} can be applied to sum a subset of diagrams up to an infinite order in a controlled manner.
% 
% In the context of non-equilibrium transport problem, Meir and Wingreen showed that the electronic current through an \textit{interacting} system can be written in terms of $A(\omega)$, the spectral function of the system. Corresponding formula for phonon transport through an anharmonic system was derived by Wang \cite{wang06} and Mingo \cite{mingo06}, and the formula reads for, say, the current flowing to the left lead
% \begin{equation}
%  I = \int \frac{d\omega}{2\pi} \omega \textrm{Tr}\left[G^R(\omega) \Sigma^<(\omega) + G^<(\omega) \Sigma^A(\omega) \right],
% \end{equation}
% where $\Sigma^<$ and $\Sigma^A$ are the lesser and advanced self-energies of the left lead. To calculate the Green's functions and self-energies perturbatively, the perturbation expansion is done for the more general Keldysh Green's function
% \begin{equation}
%  G (\tau,\tau') = -i \langle \mathcal{T}_{\tau} u(\tau) u(\tau') \rangle.
% \end{equation}
% Time variable $\tau$ lies on the Keldysh contour, which runs from $-\infty$ to $\infty$ slightly above the real axis and back to $-\infty$ slightly below the real axis \cite{jauho}.



%\begin{itemize}
% \item Definition of polarizability, optical theorem
% \item Coupling of optical and acoustic degrees of freedom
%\end{itemize}


\iffalse
\begin{equation}
 \left\langle \tilde{j}^{\alpha}(\br,\omega)\tilde{j}^{\beta}(\br',\omega') \right\rangle = 2\pi \delta(\omega+\omega') \times 2\omega \varepsilon_0 \textrm{Im}[\varepsilon(\br,\br';\omega)] \hbar \omega \left[f_B(\omega,T)+\frac{1}{2} \right]
\end{equation}
\fi
%\begin{itemize}
% \item Maxwell equations
% \item Fluctuational electrodynamics
% \item Electromagnetic Green's function
%\end{itemize}
% Loomis and Maris 94: We present a macroscopic, phenomenological theory for the heat flow between two material half-spaces of differing temperatures whose surfaces are separated by a gap of width l. Our calculation parallels Liftshitz's calculation of the van der Waals force between two dielectric slabs. For l sufficiently small, the heat flow is enhanced by a contribution from evanescent waves, and in the limit of a very small gap varies as l^{-2}.




\iffalse
Usually, the bath self-energy $\Sigma(\omega)$ is given to specify the coupling with the bath. Therefore, it is useful to derive an expression relating the bath self-energy to $M(t)$. This process is complicated by the fact that because $M(t)$ does not vanish at negative infinity, one cannot use the Fourier transform of $M(t)$ in the process. However, because only the values of $M(t)$ for $t>0$ play a role in Eq. \eqref{}, one can introduce a step-function in the integral and substitute the convenient definition $M^e(t)=\Theta(t)M(t)+\Theta(-t)M(-t)$:
\begin{equation}
 m\ddot{u}(t) =  F[u(t)] - \int_{-\infty}^{\infty}dt'\Theta(t') M^e(t')\dot{u}(t-t') + \xi(t).
\end{equation}
One can then easily show that the Fourier transform of $M^e(t)$ is related to the bath self-energy $\Sigma(\omega)$ through the coupling function $\Gamma(\omega)=-2\textrm{Im}[\Sigma(\omega)]$:
\begin{equation}
 \hat M^e(\omega) = \frac{\Gamma(\omega)}{\omega}. \label{eq:th_langevin_Mt}
\end{equation}
In cases where the exact spectral properties of the bath do not matter, the simplest choice for the bath self-energy is
\begin{equation}
 \Sigma(\omega) = -i\gamma \omega \Theta(\omega_c-|\omega|),
\end{equation}
where $\omega_c$ is the cut-off frequency for the bath modes. Equation \eqref{eq:th_langevin_Me} then gives
\begin{equation}
 \hat M^e(\omega) = 2\gamma \Theta(\omega_c-|\omega|), 
\end{equation}
so the friction kernel $M^e(t)$ is 
\begin{equation}
 M^e(t) = 2 \gamma \delta_{\omega_c}(t),
\end{equation}
where 
\begin{equation}
 \delta_{\omega_c} (t) = \frac{1}{\pi} \frac{\sin \omega_c t}{t}. \label{eq:th_langevin_deltat}
\end{equation}
For $\omega_c\to \infty$, Eq. \eqref{eq:th_langevin_deltat} tends to the Dirac Delta function and the friction term in the generalized Langevin equation reduces to the classic Langevin equation 
\begin{equation}
  m\ddot{u} = {F}[{u}(t)] -m \gamma \dot{{u}} + \xi(t). 
\end{equation}
\fi
\iffalse

\subsection{Background}
In his seminal work on the theory of Brownian motion, Paul Langevin added stochastic force terms in the equation of motion to model the essentially random collisions of a particle with the molecules of the surrounding fluid. The additional force consists of two terms, the deterministic damping force proportional to the friction coefficient $\gamma$ and the stochastic force $\xi$:
\begin{equation}
 m \ddot{x} = {F}[{x}(t)] -m \gamma \dot{{x}} + \xi(t). \label{eq:langevin_eq}
\end{equation}
Here ${x}(t)$ is the particle position, $m$ the mass and ${F}[{x}(t)]$ is the force due to particles other than the solvent. For simplicity, we have written the one-dimensional form of the equation. In Langevin theory, the collisions with the solvent (represented by the stochastic force $\xi$) are assumed to average to zero force ($\langle \xi \rangle=0$) and to be temporally uncorrelated: $\langle \xi(t) \xi(t')^T\rangle \propto \delta(t-t')$. To calculate the constant of proportionality in the variance, one can calculate the expectation value of $\langle v^2\rangle$ for $t \to \infty$ to show that the classical equipartition $ m \langle \bb{v}^2 \rangle = k_BT$ only holds if the stochastic force and friction force are related by the relation
\begin{equation}
 \langle \xi(t) \xi(t')\rangle=2\gamma T \delta(t-t'). \label{eq:corr_ohmic} %\mathbf{I}_{3\times 3}
\end{equation}
This is the fluctuation-dissipation relation connecting the magnitude of fluctuations $\xi$ to the dissipation constant $\gamma$. The damping term of Eq. \eqref{eq:langevin_eq} is often referred to as Ohmic damping due to its correspondence with an Ohmic resistor in circuit theory \cite{weiss}.

In this example, the molecules of the solvent act as a thermal reservoir at temperature $T$. For any given initial velocity of the particle, the particle will drift toward thermal equilibrium with the reservoir and eventually achieve it. Building on this idea, Langevin forces are traditionally used in simulations to thermostat the system to a given temperature \cite{}. This allows one to either (i) simulate canonical ensemble at given temperature, (ii) push the system into thermal non-equilibrium by coupling atoms to Langevin thermostats at different temperatures, or (iii) to simulate dissipative and fluctuative processes driving the system to local equilibrium by Langevin thermostats at position-dependent temperatures.
\fi
\iffalse
\section{Langevin bath in simulations}

Langevin bath is typically used for three different tasks. In the first case, Langevin bath is used to simulate canonical ensemble (thermal equilibrium) by coupling all atoms to a bath at single temperaure $T$. In this case, the coupling constant $\gamma$ to the baths should typically be chosen small enough so that the coupling does not disturb the natural vibrational dynamics in the system. If the coupling is too small, however, the energy exchange with the bath is so slow that very long simulation runs are required to properly sample the available phase space.

In the second case, multiple baths at different temperatures are used to push the system into non-equilibrium. In this case, the baths act as heat sources and sinks, and the coupling constant $\gamma$ effectively determines the contact resistance with the reservoirs. While large $\gamma$ generally decreases the contact resistance to the reservoirs, it also increases the acoustic mismatch between thermalized and unthermalized atoms. Therefore, it should be carefully checked that the obtained results (such as thermal resistance) are not sensitive to the exact value of $\gamma$.

Finally, coupling to the Langevin bath can describe \textit{internal} processes driving the system into (local) thermal equilibrium. For example, the complicated phonon-phonon interactions giving rise to phonon creation and annihilation can be described in an effective manner by the fluctuating and dissipative Langevin forces, respectively. The resulting linearization of the equations of motion allows for solving the equations of motion directly in terms of the Green's function. To ensure current conservation, it is necessary to determine the bath temperatures self-consistently so that phonon creation and annihilation are balanced. This is the self-consistent heat bath model.
\fi

\iffalse
\subsection{General Langevin equation}

The original Langevin equation with the classical fluctuation-dissipation relation is typically used in molecular dynamics simulations due to its simplicity. In cases when the spectral properties of the coupling to the bath matter (for example to minimize the contact resistance between the bath and the system) or if quantum statistics must be accounted for, one must turn to the general Langevin theory \cite{weiss}.

In general Langevin theory, the reservoir is modeled as a collection of harmonic oscillators. The bath degrees of freedom are ''integrated out'' so that their interaction with the system under study is described effectively by the Langevin forces. In general, the friction and force then have temporal correlations and the Ohmic damping of Eq. \eqref{eq:langevin_eq} and Markovian force [Eq. \eqref{eq:corr_ohmic}] are replaced by more complicated expressions \cite{weiss}. The general Langevin equation reads \cite{dhar06}
\begin{equation}
 m\ddot{u}(t) =  F[u(t)] - \int_{0}^{\infty}dt' \Sigma(t') {u}(t-t') + \xi(t),
\end{equation}
where the auto-correlation function of the random force $\xi$ is related to the damping self-energy through the fluctuation-dissipation relation
\begin{equation}
 \langle \xi(t)\xi(t') \rangle = \int_{-\infty}^{\infty} \frac{d\omega}{2\pi} e^{-i\omega(t-t')} \hbar \Gamma(\omega) [f_B(\omega,T)+1].
\end{equation}
By Fourier transforming with respect to $t$ and $t'$ separately, Eq. (XXX) can be written in the form useful for calculations:
\begin{equation}
  \langle \xi(\omega)\xi(\omega') \rangle = 2\pi\hbar\delta(\omega+\omega') \Gamma(\omega) \left[f_B(\omega,T)+1 \right].
\end{equation}
Here $\Gamma(\omega)=-2\textrm{Im}[\Sigma(\omega)]$.

Typically, the integral term in the Langevin equation is written in terms of the velocity to identify it as a frictional force. To achieve this, we integrate partially in Eq. \eqref{}:
\begin{equation}
 m\ddot{u}(t) =  F[u(t)] - \int_{0}^{\infty}dt' M(t')\dot{u}(t-t') + \xi(t)
\end{equation}
The boundary terms in the integral are assumed to vanish  because we (i) define the integral function $M(t)=-\int_t^{\infty} dt' \Sigma(t')$ of $\Sigma(t)$ so that $M(t\to \infty)=0$ and (ii) the term proportional to $M(0)u(t)$ can be absorbed to the external force $F[u(t)]$ or eliminated by re-defining the displacements \cite{weiss}. 

Usually, the bath self-energy $\Sigma(\omega)$ is given to specify the coupling with the bath. Therefore, it is useful to derive an expression relating the bath self-energy to $M(t)$. This process is complicated by the fact that because $M(t)$ does not vanish at negative infinity, one cannot use the Fourier transform of $M(t)$ in the process. However, because only the values of $M(t)$ for $t>0$ play a role in Eq. \eqref{}, one can introduce a step-function in the integral and substitute the convenient definition $M^e(t)=\Theta(t)M(t)+\Theta(-t)M(-t)$:
\begin{equation}
 m\ddot{u}(t) =  F[u(t)] - \int_{-\infty}^{\infty}dt'\Theta(t') M^e(t')\dot{u}(t-t') + \xi(t).
\end{equation}
One can then easily show that the Fourier transform of $M^e(t)$ is related to the bath self-energy $\Sigma(\omega)$ through the coupling function $\Gamma(\omega)=-2\textrm{Im}[\Sigma(\omega)]$:
\begin{equation}
 \hat M^e(\omega) = \frac{\Gamma(\omega)}{\omega}.
\end{equation}
%The fluctuation-dissipation relation then becomes
%\begin{equation}
% \langle \xi(\omega)\xi(\omega') \rangle = 2\pi\hbar\delta(\omega+\omega') \omega M^e(\omega) \left[f_B(\omega,T)+1 \right].
%\end{equation}


\subsection{Ohmic damping}
In cases where the exact spectral properties of the bath do not matter, the simplest choice for the bath self-energy is
\begin{equation}
 \Sigma(\omega) = -i\gamma \omega \Theta(\omega_c-|\omega|),
\end{equation}
where $\omega_c$ is the cut-off frequency for the bath modes. Equation (XXX) then gives
\begin{equation}
 \hat M^e(\omega) = 2\gamma \Theta(\omega_c-|\omega|), 
\end{equation}
so the friction kernel $M^e(t)$ is 
\begin{equation}
 M^e(t) = 2 \gamma \delta_{\omega_c}(t),
\end{equation}
where 
\begin{equation}
 \delta_{\omega_c} (t) = \frac{1}{\pi} \frac{\sin \omega_c t}{t}.
\end{equation}
For $\omega_c\to \infty$, Eq. (XXX) tends to the Dirac Delta function and the friction term in the generalized Langevin equation reduces to the Ohmic damping in the classic Langevin equation (XXX).
\fi
%\section{Langevin theory}

